%---------------------------------------------------------------------
%
%                      resumen.tex
%
%---------------------------------------------------------------------
%
% Contiene el cap�tulo del resumen.
%
% Se crea como un cap�tulo sin numeraci�n.
% 
% Aqu� no se pueden poner los acr�nimos como \ac{__} porque no los reconoce, 
% as� que los pongo a mano.
%
%---------------------------------------------------------------------

\chapter{Abstract}
\cabeceraEspecial{Abstract}

Corporate workers increasingly use their own devices for work
purposes, in a trend that has come to be called the \textit{Bring Your
  Own Device} (BYOD) philosophy and companies are starting to include
it in their policies. For this reason, corporate security systems need
to be redefined and adapted, by the corporate Information Technology (IT) department, to these
emerging behaviours. %adapted _by the workforce_? Why? -JJ
%creo que hace tiempo yo puse otra cosa y alguien me lo cambi� a workforce, pero lo cambio otra vez
This work proposes applying soft-computing techniques, in order to help the Chief Security Officer (CSO) of a company (in charge of the IT department) to improve the
security policies. % policies m�s que mechanisms, �no? - JJ
%cierto, lo cambio
The actions performed be company workers under a BYOD situation will be treated as events: an action or set of actions yielding to a response. Some of those events might cause a non compliance with some corporate policies, and then it would be necessary to define a set of security rules (action, consequence). Furthermore, the processing of the extracted knowledge will allow the rules to be adapted.

The goals of this work are the following:

\begin{itemize}
 \item Extracting information from
   a set % company-set security policies or the set of security
               % polices set by the company? - JJ
               % culpa m�a porque esto creo que lo puse tal cual lo pusimos en la proposici�n del TFM, lo pongo mejor
of Corporate Security Polocies (CSP) (also called Information Security Policies (ISP)), and from a set of Internet
   connection patterns of its employees, by applying Data Mining techniques. % No es un objetivo, es un
                                % medio para alcanzar un objetivo - JJ
                                % claro, el objetivo es extraer la informaci�n. Lo redacto de nuevo.
 \item Combining that information to detect interesting or suspicious
   patterns from the point of view of a CSO. They are relevant because they could lead to security incidents, assuming that a security incident is a pattern that is not compliant with the rules derived from security policies. %interesting
                                %for whom? - JJ
                                %creo que est� bastante claro si se dice que pueden dar lugar a incidentes de seguridad...
                                %pero incluyo lo del CSO para m�s aclaraci�n.
 \item Performing a deep study of the labelled patterns with the Weka
   tool, and find the best classification method and the most
   significant data features. % features of what? - JJ
                              % cambio 'features' por 'data features'
\end{itemize}

All the development of the code, research, and writing of this Thesis has been done on Github \footnote{\url{https://github.com}}. It is open and accesible at: \url{https://github.com/unintendedbear/TFM/}.

\endinput
% Variable local para emacs, para  que encuentre el fichero maestro de
% compilaci�n y funcionen mejor algunas teclas r�pidas de AucTeX
%%%
%%% Local Variables:
%%% mode: latex
%%% TeX-master: "../Tesis.tex"
%%% End:

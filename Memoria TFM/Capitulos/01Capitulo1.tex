%---------------------------------------------------------------------
%
%                          Chapter 1
%
%---------------------------------------------------------------------
\renewcommand{\figurename}{Figure}
\chapter{Introduction}

\begin{FraseCelebre}
\begin{Frase}
Ha, ha! Whoo hoo hoo! Ah! Geronimo!
\end{Frase}
\begin{Fuente}
11th Doctor. Doctor Who. \textit{The End Of Time, part II}.
\end{Fuente}
\end{FraseCelebre}


%-------------------------------------------------------------------
\section{Introduction}
%-------------------------------------------------------------------
\label{cap1:sec:intro}

The way in how data has been stored and accessed in the companies has
completely changed over the last few
years \citep{data_infographic}. %use
                                %uniform style for articles, that is,
                                %bibtex. Footnotes are for aclarations
                                %- JJ
                                % fixed
On the one hand, start-up companies get a great benefit either by using services of the cloud \citep{Vaqu_Cloud09}, or by offering their services through it \citep{Leav_Cloud09}. For bigger companies, on the other hand, the use of company-owned servers being accessed from desktop PCs (and maybe laptops) from within the company facilities has been transformed. Now, these data are distributed among a number of machines, even not all bemacselonging to the company, and working over a cloud Computing environment.
%why Cloud in Caps? - JJ
%fixed
Besides, being stored in the cloud or not, data are being consulted and modified through a wide amount of devices, some of them owned by the company's users. This is the so-called \ac{BYOD} \citep{Cald_byod12} philosophy, which also means that people do not use their smart devices for one purpose only (personal life or business) anymore. It is becoming highly successful due to the impact that portable devices (such as smartphones and tablets) are having in the market. And because of data security and privacy are key factors for a company, it is usual to define Organisational Security Policies in order to ensure them \citep{Blaze_SecPolic99}. Their definition is nowadays a very difficult problem, since that \ac{BYOD} tendency means that several factors must be considered \citep{Opp_Security11}, most of them previously ignored or non considered in security systems, for instance the current mixture between personal and professional information in these devices (the user could navigate inside social networks where there could be friends and also company partners or clients).

%-------------------------------------------------------------------
\subsection{Enterprise security}
%-------------------------------------------------------------------
\label{cap1:sec:intro:subsec:enterprise}

Now that corporate networks are becoming dynamic for being adapted to the \ac{BYOD} philosophy, there is an additional risk because the devices that the employees use are not always company-owned. Thus, several solutions have arisen in order to manage the corporate security in a \ac{BYOD} scenario. Some of them are focused only on smartphones, while others are thought for laptops; also some are implemented for a certain platform, but others consider multiplatform. However most of them try to be non-intrusive (regarding the users' personal data), friendly and easy to use. In Section \ref{cap2:sec:tools} we present an overview of the main solutions, describing their features, advantages and flaws.

This kind of monitorization and security-aimed applications are being developed for many reasons like the aforementioned, and also because it has been demonstrated that people are the main hazard regarding the company security \citep{Adams_users99}. With a methodology of work like the one described, users are allowed to start or continue a working session over multiple devices and locations without any significant loss of data. This new situation has a big impact from the point of view of the security \citep{Schum_SecPat05}, since company data borders have changed in the last years so now the users can access significant data from outside the enterprise, and possibly through a non absolutely secure channel.

Another way of protecting an enterprise is by means of a security policy, or in this case, a set of \ac{ISP}, which should deal with the way of protecting a certain organization information against a security breach. Though there are standards, such as the ISO27002 or the Security Forum's Standard of Good Practice \footnote{https://www.securityforum.org}, and many guidelines \citep{Wood_SecPol09}; an \ac{ISP} is built depending on the characteristics of the community/organisation that they are thought for.

Then, another issue to cope with is the elaboration of a good \ac{ISP}, understandable for every user of the company, and more importantly, non-intrusive for him. A lot of researchers have studied the natural tendency of employees to comply with the \ac{ISP} \citep{Sip_SecPriv07,Bulg_SecPol10,AlOmari_SecPol12}, reaching conclusions such as the employees compliance with the security policies increases educating/training them in information security awareness  \citep{Shaw_SecAware09}, and decreases applying too much sanctions when a misuse or abuse occurs \citep{Her_SecPol09}.

Normally, the enterprise network architecture was being adapted to cope with external attackers \citep{MIT_05}. With the incorporation of \ac{BYOD}, the threat is about corporate assets being compromised due to employees' devices with vulnerabilities \citep{Orth_Android12}, or leaked because they are being accessed from a device connected through an unsecure (public) network.

Thus now, more things should be considered than the usual ones when
designing a company network architecture. In Figure
\ref{fig:proposed_diagram} there is an approach to the system architecture of a secure
% too many loops and speak straight: this figure shows
% an approach to the system architecture of... - JJ
%fixed
environment. It includes the possibility of having employee-owned
mobile (smartphones and tablets) and portable (laptops) devices, and
also the opportunity that the employees have of connecting these
devices either from inside or outside the company premises. Moreover,
company's information assets are constantly accessed under these
conditions, considering that an information asset means every
\textit{piece of information} that has a \textit{value} (cost
depending on the risk of being lost or leaked) for the company. It can
be referred to files with sensitive information to certain mails, or
even to company applications. 

\figuraEx{Vectorial/proposed_diagram.pdf}{width=0.9\textwidth}{fig:proposed_diagram}%
{Architecture approach of an Enterprise Network assuming that the Company has adopted the \ac{BYOD} philosophy.}{Architecture approach of an Enterprise Network assuming that the Company has adopted the \ac{BYOD} philosophy.}

This situation leads to a need of protecting the organisation's side, but also the users' side, making non-interfering easy-to-follow \ac{ISP}s, and leaving them to use their devices for personal purposes while working, without putting organisation's information assets under risk. The compliance of these requirements would compose an end-to-end security solution (protecting both enterprise and employee).

%-------------------------------------------------------------------
\section{Objectives}
%-------------------------------------------------------------------
\label{cap1:sec:intro:subsec:objectives}

The problem to solve is related with the application of corporate security policies in order to deal with potentially dangerous \ac{URL} accesses inside an enterprise. Not only there could be connections to non-confident (or non-certified) web sites (referenced by their URLs in this work), but in a company, several web pages might be also controlled for productivity or suitability reasons. Thus, an \ac{ISP} normally define sets of allowed or denied pages/websites that could be eventually accessed by enterprise employees. These sets are usually included in a White (permitted) or Black (non-permitted) Lists. These lists act as a good control tool for those \ac{URL} included in them as well as for the complementary, i.e. \ac{URL}s that are not included in a Whitelist have automatically denial of access, for instance.

In this work we go a step beyond, trying to define a tool for automatically making an allowance or denial decision with respect to \ac{URL}s that are not included in the aforementioned lists. This decision would be based in that one made for similar \ac{URL} accesses (those with similar features), but considering other parameters of the request/connection instead of just the \ac{URL} string, as those lists do.

Therefore, the starting point is: to have a registration of the Internet accesses made in a company, and a set of policies that say what should or should not be done.

The first objective of this work is to be able to label the whole set of entries in the access log, by means of simply applying the rules to them. To this end a dataset of \ac{URL} sessions (requests and accesses) should be analysed. These data are labelled with the corresponding permission or not for that access following the aforementioned policies. The problem is then transformed into a classification one, in which every new \ac{URL} request will be classified, and thus, a grant or deny action will be assigned to that pattern.

The desired outcome is to obtain as many labelled entries as possible, to proper build and train a classifier. We will have, then, labelled entries as \textit{allowed} or \textit{denied} by the company, and unlabelled entries.

The next objective is to obtain the best classification accuracies as possible, so a series of experiments should be performed, in order to see which classifiers work better. Also, it is important to take into account that balancing techniques may be applied if the dataset present imbalance \citep{imbalance_techniques_02}.

Finally, with a trained classifier, those patterns that were unlabelled, are expected to be now labelled (at least, most of them). Therefore, a system that adopts that classifier, would be able to adapt to new situations and new policies could be written for coping with them.

The Thesis is structured as follows. Next Chapter (Chapter
\ref{cap2:SotA}) describes the state of the art of applications or
devices related to \ac{BYOD} being adopted by companies, and
protection of the users' privacy. Also, it describes related work with
regard to the application of Data Mining and Machine Learning
techniques to security issues inside a company and, finally,
introduces some work review about the study of \ac{URL}s and its use
for security menaces detection. 
In Chapter \ref{cap3:data}, we recall again the problem this work tries to solve and describe the type of policies and rules used to label the data, as well as the datasets we have worked with. The followed methodology is described in Chapter \ref{cap4:methodology}, concerning the data preprocessing and the implementations done in Perl and Java languages. Chapter \ref{cap5:results} is devoted to present the results. From a first round of experiments comparing different classification methods, to once the best of them are selected, the set of conducted experiments, the different partitions of the datasets, and result tables. Finally, the conclusions and future lines of research are presented in Chapter \ref{cap6:conclusions}.


% Variable local para emacs, para  que encuentre el fichero maestro de
% compilaci�n y funcionen mejor algunas teclas r�pidas de AucTeX
%%%
%%% Local Variables:
%%% mode: latex
%%% TeX-master: "../Tesis.tex"
%%% End:

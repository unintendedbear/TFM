%---------------------------------------------------------------------
%
%                          Chapter 1
%
%---------------------------------------------------------------------

\chapter{Introduction}

\begin{FraseCelebre}
\begin{Frase}
...
\end{Frase}
\begin{Fuente}
...
\end{Fuente}
\end{FraseCelebre}

\begin{resumen}
...
\end{resumen}


%-------------------------------------------------------------------
\section{Introduction}
%-------------------------------------------------------------------
\label{cap1:sec:intro}

...

%-------------------------------------------------------------------
\subsection{Objectives}
%-------------------------------------------------------------------
\label{cap1:sec:intro:subsec:obj}

The way in how data has been stored and accessed in the companies has completely changed over the last few years\footnote{http://mozy.com/infographics/the-past-present-and-future-of-data-storage/}. On the one hand, start-up companies get a great benefit either by using services of the Cloud \citep{Vaqu_Cloud09}, or by offering their services through it \citep{Leav_Cloud09}. For bigger companies, on the other hand, the use of company-owned servers being accessed from desktop PCs (and maybe laptops) from within the company facilities has been transformed. Now, these data are distributed among a number of machines, even not all belonging to the company, and working over a Cloud Computing environment.

Besides, being stored in the Cloud or not, data are being consulted and modified through a wide amount of devices, some of them owned by the company's users. This is the so-called \ac{BYOD} \citep{Cald_byod12} philosophy, which also means that people do not use their smart devices for one purpose only (personal life or business) anymore. It is becoming highly successful due to the impact that portable devices (such as smartphones and tablets) are having in the market. And because of data security and privacy are key factors for a company, it is usual to define Organisational Security Policies in order to ensure them \citep{Blaze_SecPolic99}. Their definition is nowadays a very difficult problem, since that \ac{BYOD} tendency means that several factors must be considered \citep{Opp_Security11}, most of them previously ignored or non considered in security systems, for instance the current mixture between personal and professional information in these devices (the user could navigate inside social networks where there could be friends and also company partners or clients).

In this scenario several solutions have arisen in order to manage the corporate security. Some of them are focused only on smartphones, while others are thought for laptops; also some are implemented for a certain platform, but others consider multiplatform. However most of them try to be non-intrusive (regarding the users' personal data), friendly and easy to use. In Section \ref{cap2:sec:SotA:subsec:tools} of Chaper \ref{cap2:sec:SotA} we present an overview of the main solutions, describing their features, advantages and flaws.



%-------------------------------------------------------------------
\section*{\NotasBibliograficas}
%-------------------------------------------------------------------
\TocNotasBibliograficas

Citamos algo para que aparezca en la bibliograf�a\ldots
\citep{ldesc2e}

\medskip

Y tambi�n ponemos el acr�nimo \ac{CVS} para que no cruja.

Ten en cuenta que si no quieres acr�nimos (o no quieres que te falle la compilaci�n en ``release'' mientras no tengas ninguno) basta con que no definas la constante \verb+\acronimosEnRelease+ (en \texttt{config.tex}).


%-------------------------------------------------------------------
\section*{\ProximoCapitulo}
%-------------------------------------------------------------------
\TocProximoCapitulo

...

% Variable local para emacs, para  que encuentre el fichero maestro de
% compilaci�n y funcionen mejor algunas teclas r�pidas de AucTeX
%%%
%%% Local Variables:
%%% mode: latex
%%% TeX-master: "../Tesis.tex"
%%% End:

%---------------------------------------------------------------------
%
%                          Chapter 5
%
%---------------------------------------------------------------------

\chapter{Results}
\label{cap5:results}

\begin{FraseCelebre}
\begin{Frase}
You know the Doctor. You understand him. You will predict his actions.
\end{Frase}
\begin{Fuente}
DALEK. Doctor Who. \textit{The Parting of the Ways}.
\end{Fuente}
\end{FraseCelebre}

%-------------------------------------------------------------------
\section{Labelling Process and discussion}
%-------------------------------------------------------------------
\label{cap5:sec:labels}

%False positives/Negatives

%-------------------------------------------------------------------
\section{Experiment results}
%-------------------------------------------------------------------
\label{cap5:sec:exp}

All the result tables are publicly accessible through the following link: \url{https://drive.google.com/folderview?id=0B0Rvxfh8NwQRMi1RYU9uT3FmQzg&usp=sharing}






%-------------------------------------------------------------------
\section*{\NotasBibliograficas}
%-------------------------------------------------------------------
\TocNotasBibliograficas

Citamos algo para que aparezca en la bibliograf�a\ldots
\citep{ldesc2e}

\medskip

Y tambi�n ponemos el acr�nimo \ac{CVS} para que no cruja.

Ten en cuenta que si no quieres acr�nimos (o no quieres que te falle la compilaci�n en ``release'' mientras no tengas ninguno) basta con que no definas la constante \verb+\acronimosEnRelease+ (en \texttt{config.tex}).


%-------------------------------------------------------------------
\section*{\ProximoCapitulo}
%-------------------------------------------------------------------
\TocProximoCapitulo

...

% Variable local para emacs, para  que encuentre el fichero maestro de
% compilaci�n y funcionen mejor algunas teclas r�pidas de AucTeX
%%%
%%% Local Variables:
%%% mode: latex
%%% TeX-master: "../Tesis.tex"
%%% End:

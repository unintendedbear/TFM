%---------------------------------------------------------------------
%
%                          Chapter 2
%
%---------------------------------------------------------------------

\chapter{State of the Art}
\label{cap2:SotA}

\begin{FraseCelebre}
\begin{Frase}
We both contain the knowledge of over nine hundred years of memory and experience. 
\end{Frase}
\begin{Fuente}
11th Doctor. Doctor Who. \textit{The Almost People}.
\end{Fuente}
\end{FraseCelebre}

The present Master Thesis tries to obtain a URL classification tool for enhancing the security in the client side, as at the end we want to get if a certain URL is secure or not, having as reference a set of rules (derived from a CSP) that allow or deny a set of known \textit{http} requests. For this, \ac{DM} and \ac{ML} techniques have been applied. Also, this solution tends to be included inside a \ac{BYOD} environment as described at the beginning of the Section \ref{cap1:sec:intro:subsec:obj}. This Chapter gives an overview in a number of solutions given to protect the user, or the company, against unsecure situations.

%-------------------------------------------------------------------
\section{Tools for corporate mobile security}
%-------------------------------------------------------------------
\label{cap2:sec:tools}

Now that \ac{BYOD} philosophy is becoming a tendency, a number of tools appeared by now, and others are under development but expected to be released soon. They were specifically designed for \ac{CSO}s and \ac{CISO}s to secure, monitor, and act over spartphones and other personal mobile or portable devices, when they become too risky.

%Extender

%-------------------------------------------------------------------
\subsection{IBM Hosted Mobile Device Security Management}
%-------------------------------------------------------------------
\label{cap2:sec:tools:subsec:ibm}

One of the first companies who supported the \ac{BYOD} model was IBM in 2011 \citep{IBM_tool}, as they recognized the increase of employees who brought their personal smartphones or tablets into the workplace. To help organizations \citep{Kao_ibm11} embracing both company and employee owned mobile devices (as said, this practise is part of the \ac{BYOD} model) in a security-rich environment, IBM developed a mobile device security management solution. For IBM, a mobile security strategy should focus on several key areas.

Thus, the organization should identify which business data the strategy will allow to be stored and processed on which mobile devices. This helps to determine what needs to be protected and to what degree. Then, since different mobile platforms have different native security mechanisms, the organization needs to define which mobile device platforms will be allowed in the business environment and, thus, which need to be supported in the mobile security strategy and plan. That means, to define the scope of the security plan. Also it is needed to decide the responsibility for mobile security management work, whether using the current \ac{IT} security team to handle mobile devices, or outsourcing them to a managed security service provider. And no matter what the mobile environments are, a number of mobile ISPs and best-practise procedures need to be put in place and should also be identified in the company's mobile security strategic plan.

Taking into account these considerations, IBM developed a framework that specifies security domains and levels for applying various security technologies. When applied to mobile devices (not portable ones, so that laptops are not considered), the features of this framework depend on the deployment:

\begin{description}
	\item[Identity and access] Not only the use of strong passwords when accessing the devices, but also supply two way authentication and \ac{VPN} access control, by supervising the authorised \ac{IP} addresses and adding re-authentication for accessing resources (assets) with high value.
	\item[Data protection] Data security stored in the devices, related to professional life, is encrypted, also during the transmission. Moreover, when a device is lost, data can be removed remotely, and the device located or locked out (locking out a device is also made after a modifiable timeout). Finally, back ups are performed periodically once the device has been recovered.
	\item[Application security] This is related to the requirement of being connected from a controlled location for authorising the download of business applications, and those applications must be certified (also the currently running on the device). Actually, IBM's tool can monitor and remove applications which are identified as untrustworthy or unsafe.
	\item[Fundamental integrity control] Running both anti-malware software and personal firewall, and making the result of these tests as dependency for \ac{VPN} intranet access. 
	\item[Governance and compliance] Taking into consideration the mobile security as an important point in the overall risk management program of the company, and extend the periodic security audits to mobile devices too.
\end{description}

The considered architecture for this solution is a client-server architecture, where the controlling center is placed in the server, and the client would be installed on the mobile devices. This way, the client/device would communicate with the server as regularly as possible, to enforce policies, execute commands, and report status. With these features, and by analysing in the server the incoming data from the devices, this IBM's solution would be able to help enterprises to support \ac{ISP} compliance and to recognise (consequently acting over) mobile threat landscapes.

%-------------------------------------------------------------------
\subsection{Sophos Mobile Control}
%-------------------------------------------------------------------
\label{cap2:sec:tools:subsec:sophos}

Sophos is a company founded in 1985 focused on \ac{IT} security and data protection for businesses. Their \textit{Mobile Device Management} main product \citep{Sophos_tool} is Sophos Mobile Control. It is oriented to \ac{IT} administration for mobile devices, trying to offer to the users the possibility of choosing the delivery model to suit their needs, i.e., between on-premise and \ac{SaaS}.

The tool offers the possibility of managing all workers and co-workers smartphones and tablets from a single-based console. The console monitors the devices throughout their full life cycle: from the initial set up and enrolment, right through to decommissioning. Other features are similar to IBM's product, adding some new like being able to connect to an existing user directory using \ac{LDAP}\footnote{\ac{LDAP} is an application protocol for accessing and maintaining distributed directory information services over an \ac{IP} network.}.
	
Additional security is provided by the incorporation of \textit{Malware and Web protection}, so that the user does not need to install anti-malware software by him or herself. Regarding the compliance enforcement, the main goal is not to sacrifice company's security in favour of flexibility for the users, which has been demonstrated \citep{Her_SecPol09} that could result in bad (unwilling or not) users' behaviour. Thus, company's \ac{BYOD} initiative should include an acceptable use policy to ensure the users are aware of any measures the company may take if a device breaches any \ac{ISP}s. Sophos aims to reach this by doing three main tasks: First, by enforcing \ac{ISP}s, i.e. allowing setting up user and group-based security policies separately. The security settings can also vary from one platform to another, thus looking to support all of them is necessary to set task bundles and individual actions for many different violations. Secondly, by a risk mitigation in which the actions to perform can be set according to the severity of a breach. For minor cases, the company may want to simply inform to the user, but if sensitive data is at risk, a remote wipe may be the chosen option. As previously stated, the actions vary for each platform, but the most common platforms such as Android and iOS allow blocking email access, notifying the admin, performing a remote lock or wiping, locating a device using 3D maps, triggering a remote alarm, transferring a task bundle combining a number of actions, and Sophos's solution adds the possibility of trigger a scan. Finally, compliance check, i.e. though some of the most widely used features include allow or disallow root rights or jailbreaking, require encryption, and whitelist or blacklist apps. Sophos also allows disabling malware apps, setting maximum intervals since last \textit{Mobile Security scan}, and allowing or disallowing suspicious apps and \ac{PUA}s.
	
Then, the Enterprise App Store included in Sophos Mobile Control allows the company to supply the users with recommended and/or required apps directly on their devices. Both company's in-house and app store apps are suggested directly on the user's mobile device, so they can click to trigger the installation. Also, for keeping the employees working without increasing the burden for the IT department, the self-service portal built-in included in Sophos's solution has many features. Among others, it shall be mentioned that, wanting the employees to use their personal devices at work, they can register them (with a provided step-by-step process) and agree to an acceptable company-defined use policy. All profiles, including email access, would be available after registration, and the portal may be accessed from any PC with Internet access, or even from a mobile device itself. Furthermore, when a device is eventually stolen, users can choose to remotely locate, lock or wipe their devices and reset their passcode without having to contact the company help desk. From the company side, they can define which features are available in a self-service portal from the administrator console.

%-------------------------------------------------------------------
\subsection{Samsung's Knox Mobile Security Suite}
%-------------------------------------------------------------------
\label{cap2:sec:tools:subsec:knox}

As part of its SAFE (Samsung for enterprise) brand, Samsung revealed at the Barcelona Mobile World Congress 2013 \footnote{http://www.zdnet.com/mwc-2013-samsungs-knox-system-takes-byod-fight-to-blackberry-7000011770/} the Knox application \citep{Samsung_tool}, and was publicited at the congress again this year \footnote{https://www.samsungknox.com/es/blog/new-knox-solutions-being-announced-mwc}. The solution is available since December 2013.
The main feature of this security package is the use of different containers, or environments, for business and personal sides. Each one even includes its own graphic configuration (wallpapers, colours, and so on), in order to be more easily recognised and distinguished by the user.
There will be needed introducing a password to enter into the business side and, once \textit{logged in} this container, no more passwords will be required for the business applications. The applications approved by the company \ac{IT} department must meet Samsung's security standards and allow single sign-on. A Knox \ac{API} will also be provided for the company to be capable of accessing over almost 205 predefined IT policies (the SAFE \ac{API} grows this number to 475). Also, Knox would allow different VPNs for individual apps.
Regarding the information protection methods, data files saved by applications of each environment are encrypted with AES 256-bit algorithm, in such manner that only the appropriate container can access these files. In the same way, the user won't be able to share data between the two environments, e.g. creating separate contact lists so the user cannot send a contact from one side to the other, or if the user copies data to the clipboard in the Knox container, it won't be there in the personal container. Figure \ref{fig:img_knox_01} shows the device architecture with three main parts, each one in charge of deploying some of the mentioned characteristics.

\figuraEx{Vectorial/samsung_knox.pdf}{width=0.8\textwidth}{fig:img_knox_01}%
{Samsung's Knox utility architecture. Source: \url{http://www.samsung.com/global/business/mobile/solution/security/samsung-knox}.}{}

\begin{description}
	\item[Customizable Secure Boot] This ensures that only verified and authorized software can run on the device. It is a primary component that forms the first line of defence against malicious attacks on devices with Samsung Knox. In addition, Samsung Knox's Secure Boot technology allows the switch of the secure boot root certificate in a secure manner after the devices are shipped.
	\item[TrustZone-based Integrity Measurement Architecture (TIMA)] By a continuous integrity monitoring of the Linux kernel, it is possible to detect that the integrity of the kernel or the boot loader is violated, and to take a policy-driven action in response. One of these policy actions disables the kernel and powers down the device.
	\item[Security Enhancements for Android] This feature is the one referred to the separation of information based on confidentiality and integrity requirements. It isolates applications and data into different domains so that threats of tampering and bypassing of application security mechanisms are reduced while the amount of damage that can be caused by malicious or flawed applications is minimized.
\end{description}

It is important to take into account that if an enterprise decides to deploy this solution to secure its BYOD environment, it must yet work with certain \ac{MDM} services, either cloud or server based. Samsung offers a list\footnote{https://www.samsungknox.com/en/knox-mdm-feature-list} of the MDM which Knox supports, each one with the number of supported policies.

%-------------------------------------------------------------------
\subsection{Good's Bring Your Own Device solution}
%-------------------------------------------------------------------
\label{cap2:sec:tools:subsec:good}

The philosophy followed by Good's solution is similar to Samsung's Knox one: to create a secure container that places an unreachable partition between personal and business data in order to protect company assets. The solutions that they offer \citep{Good_tool} are similar than the previous ones. They have focused in mobile (not laptops) devices, though they support a different of \ac{OS}s, and in separating personal and company data. A Good's secure Network Operations Center (NOC) is introduced for dealing with the unauthorised devices, or for providing access to secure collaboration solutions (email, PIM, calendar), intranet, and in-house or third-party mobile applications. Finally, Good offers best practise recommendations to help the company's \ac{BYOD} policies such as reimbursements and stipends. There is a document available at Good's webpage\footnote{http://www1.good.com/mobility-management-solutions/bring-your-own-device}  which contains several questions about \ac{ISP}s and how to cope with all of them.
                                
%-------------------------------------------------------------------
\subsection{BlackBerry Balance}
%-------------------------------------------------------------------
\label{cap2:sec:tools:subsec:bb}

This security package was announced as a feature of BlackBerry 10 \citep{Blackberry_tool}. Nevertheless, it is available with BlackBerry Enterprise Service 10, which is a device management, security and app management for BlackBerry, iOS and Android devices. It is necessary to activate BlackBerry Balance for having available some security features, all related or similar to the aforementioned. For instance, as shown in Figure \ref{fig:blackberry_bal}, a message is displayed when the user tries to copy work data and then paste it into personal apps. Also, user attempting for actions that are not permitted in the company \ac{ISP}, or may cause secure work information to be in contact with personal applications, these actions won't be permitted. 

\figuraEx{Vectorial/Blackberry_balance.pdf}{width=0.5\textwidth}{fig:blackberry_bal}%
{Blackberry Balance. Source: http://uk.blackberry.com/business/software/blackberry-balance.html.}{Displayed message in new Blackberry 10 when attepmting to copy sensitive company data}

On the other side, employees are able to access information and applications related to their personal lives, while staying connected to important work information when they need to perform. Finally, another known feature is also offered by Blackberry, so if the device gets lost or is stolen, or if the employee leaves the organization, there will be an option to wipe just work information which can be done remotely.

%-------------------------------------------------------------------
\subsection{Blackphone}
%-------------------------------------------------------------------
\label{cap2:sec:tools:subsec:blackphone}

One of the most acclaimed devices that were presented in this year Mobile World Congress (Barcelona, Spain), was the Blackphone \footnote{\url{http://www.pcmag.com/article2/0,2817,2453964,00.asp}}. It is a phone developed with the objective of keeping the user safe at all moments, so it is a perfect match for a BYOD ecosystem. The data stored on the device can be wiped remotely at any time, and the use of the Blackphone Security Center is really helpful for the user, as it is possible completely control every application that is installed.

Blackphone's Operative System, called PrivateOS, is a version of Android customised to offer the maximum security, and it has the hability, for example, of ignoring any Wi-Fi spot except for those that are trusted by the user. In Figure \ref{fig:blackphone} there is a list of the main advantages and security and privacy enhancements that PrivatOS has with respect to a standard Android OS.

\figuraEx{Vectorial/blackphone.pdf}{width=1.\textwidth}{fig:blackphone}%
{Blackphone's PrivatOS Features. Source: https://www.blackphone.ch}{The features are compared with the ones in Android from the point of view os security and user privacy.}

%-------------------------------------------------------------------
\subsection{Citrix}
%-------------------------------------------------------------------
\label{cap2:sec:tools:subsec:citrix}

XenMobile with Worx App SDK, made by Citrix Systems, provides BYOD security services to companies using fine-grained policies to prevent users from performing unallowed actions (like using the mobile phone's camera, GPS or microphone) \citep{citrix_13}. These policies can be turned on or off using its own GUI. Citrix is framework-enabled, and it is aware of some (or all) apps installed on the device. All the apps that are Worx-enable are capable of interacting, and thus offering the user a better experience. 

It is also important to point out that Citrix differenciates between apps used by the user privately and those used for business, locating both on a secure mobile container that is encrypted, and can be locked remotely for safety reasons. Another feature of Citrix is that it uses dedicated micro VPN to connect to Citrix-protected backend services.

%-------------------------------------------------------------------
\subsection{WSO2 Enterprise Mobility Manager}
%-------------------------------------------------------------------
\label{cap2:sec:tools:subsec:wso2}

WSO2 Enterprise Mobility Manager (WSO2 EMM) \citep{wso2_tool} is an open source platform that also works with the BYOD program. Some of this WSO2 EMM key features are:

\begin{description}
  \item[Mobile Device Management] that is used to manage both user and corporate owned devices, providing support for Android and iOS at the moment. It should be noted that this tool allowes the tracking of every enrolled device, as well as obtaining reports and analytics of their use.
  \item[Mobile Application Management] With regard to the software, this platform is able to allow or deny the use of applications on enrolled devices based on the role of the user or policies, thus restricting the use of some apps to certain users.
  \item[Enterprise App Store] This store provides users with both enterprise and public applications approved by the company.
  \item[Mobile Data Security] WSO2 also allows the user's data to be encrypted via password.
\end{description}

%-------------------------------------------------------------------
\subsection{Azzurri Icon Mobile Device Management}
%-------------------------------------------------------------------
\label{cap2:sec:tools:subsec:azzurri}

ICON Mobilise is a Managed Service that enables organisations to secure, manage and maximise their mobile devices and application. Azzurri \citep{azzurri_tool} offers ICON (Intelligent Cloud Optimised Network) as a cloud mobile management tool for Blackberry, Windows Phone, Android and iOS.

Although Blackphone's main feature was his ultra-secure customised OS, ICON centrally deployes, administers and secures all mobile devices regardless of type or OS.

%-------------------------------------------------------------------
\subsection{Oesis Framework}
%-------------------------------------------------------------------
\label{cap2:sec:tools:subsec:oesis}

OESIS Framework \citep{oesis_tool} is a cross platform, open development framework that enables software engineers and technology vendors to develop products that detect, classify and manage thousands of third-party software applications. The extensive functionality of this robust framework gives solutions the ability to perform detailed endpoint assessment and management on Windows, Mac, Linux, and mobile devices.

%-------------------------------------------------------------------
\section{Extracting knowledge from Data}
%-------------------------------------------------------------------
\label{cap2:sec:data}

Due to the nature of the data (URL accesses performed by humans), the used set of data is highly unbalanced \citep{imbalanced_data_05}. In order to deal with this problem there exist several methods in the
literature, but all of them are mainly grouped in three techniques
\citep{imbalance_techniques_02}: 

\begin{itemize}
 \item \textit{Undersampling the over-sized classes}: i.e. reduce the considered number of patterns for the classes with the majority.
 \item \textit{Oversampling the small classes}: i.e. introduce additional (normally synthetic) patterns in the classes with the minority.
 \item \textit{Modifying the cost associated to misclassifying the positive and the negative class} to compensate for the imbalance ratio of the two classes. For example, if the imbalance ratio is 1:10 in favour of the negative class, the penalty of misclassifying a positive example should be 10 times greater.
\end{itemize}

The first option has been applied in some works, following a random undersampling approach \citep{random_undersampling_08}, but it has the problem of the loss of valuable information. 

The second has been so far the most widely used, following different approaches, such as SMOTE (Synthetic Minority Oversampling Technique) \citep{smote_02}, a method proposed by Chawla et al. for creating `artificial' samples for the minority class, in order to balance the amount of them with respect. However this technique is based in numerical computations, which consider different distance measures, in order to generate useful patterns  (i.e. realistic or similar to the existing ones).

The third option implies using a method in which a cost can be associated to the classifier accuracy at every step. This was done for instance by Alfaro-Cid et al. in \citep{cost_adjustment_07}, where they used a \ac{GP} approach in which the fitness function was modified in order to consider a penalty when the classifier makes a false negative (an element from the minority class was classified as belonging to the majority class).
However almost all the approaches deal with numerical (real, integer) data.

One interesting point about URL classification is that the study of the distance between \ac{URL}s may be based in the distance between two strings, but Blanco et al. \citep{Blanco2011} argues that the lexical distance between two URLs is not enough to classify them. In addition, the heuristic study of \ac{URL}s for security purposes in the user side is not a novel practice. Also, the use of Blacklists (in this work, the \textit{denied} \ac{URL}s) and Whitelists (\textit{allowed} URLs) are very extended practices. For instance, phishing is a problem of security that Sheng et al.  and Khonki et al. \citep{Khonji2011} tried to solve. The first work uses Blacklists as reference to avoid phishing attacks made by e-mail; the second one aims for an heuristic analysis of the \ac{URL}s domain names and its ranks, in a way that a phished \ac{URL} can be detected.

Also, doing some web searching we have found that a lot of companies stands for the use of one between Blacklist and Whitelist \footnote{http://kevtownsend.wordpress.com/2011/08/24/whitelisting-vs-blacklisting/}. While whitelisting is the more restrictive solution and therefore the more secure, we think that the best solution is to use both, and for this reason the set of rules that we used covers a succession of either allowed and denied web sites.

What refers to the used techniques, \ac{DM}, as well as \ac{ML}, has been used since long ago in many scientific fields,
and given that research in computer security was growing since the eighties \citep{computer_security_80}, it was in the nineties when these techniques began to be applied to security issues \citep{Clifton1996}. 

On the one hand, \ac{DM} helped to develop new solutions to computer forensics \citep{DeVel2001}, being the researchers able to extract information from large files with events gathered from infected computers. Another important advance took place after the 9/11 events, when \textit{clustering techniques} and \textit{social network analysis} started to be performed in order to detect pontential crime networks \citep{Hsinchun2003}.
On the other hand, and more focused on the user side like our approach, there exist some user-centric solutions to problems like user authentication in a personal device, who Greenstadt and Beal \citep{cognitive_security_08} proposed to address using collected user biometrics along with machine learning techniques.
 
Then, when a \ac{ISP} is going to be applied, P.G. Kelley et al. \citep{user-controllable_learning_08} found important to include the user in the machine learning process for refining the policy model. They called it \textit{user-controllable policy learning}. Another approach to the refinement of user's privacy policies has been described by Danezis in \citep{inferring_policies_socialnetworks_09}, for he uses ML techniques over the user's settings in a social network, being capable of restricting permissions to other people depending on their interaction with the user.

In the same line, Lim et al. propose a system \citep{sec_policy_evolution_gp_08,pol_evol_gp_3_approaches_08} that evolves a set of computer security policies by means of \ac{GP}, taking again into account the user's feedback. Furthermore, Suarez-Tangil et al. \citep{rule_generation_gp_09} take the same approach as Lim et al., but also bringing event correlation in. These two latter author's works are interesting for ours, though they are not focused on company \ac{ISP}s - for instance, our case with the allowed or denied http requests -.

%-------------------------------------------------------------------
\section{URL filtering}
%-------------------------------------------------------------------
\label{cap2:sec:url}

As in this work we are dedicating an important part to \ac{URL}, its different parts and the way they influence in the classification process, this section is devoted to review some works related with the study of \ac{URL}s. In particular, we found interesting those works that try to identify malicious sites (like the ones that want to perform a pishing attack). It is also interesting if they study the \ac{URL} lexical features, like the one performed during this Master Thesis, and because we consider this kind of study better than to download and proccess the page (the thing that in fact is trying to be avoided). 

Hence, Kan and Thi \citep{Kan_URL05} focus their work in lexical features in order to classify as dangerous, \ac{URL}s that were not previously in Blacklists servers. They gather features like the URI components, length, ortographic data, or segments by entropy reduction. Their results are close to 95\% of accuracy.

On the other hand, this work not only focuses in lexical features of the requested \ac{URL}, but also in other data that appears in the log files. And not exactly log data but Zhang et al. \citep{Zhang_cantina07}, with CANTINA, detect pishing \ac{URL}s by studying lexical features, content related features, and a WHOIS query (obtaining the date when the domain were registered, which if it is too new, it can be less trustful). They also obtain a 95\% of accuracy.

The most important work we have found related to this were of J. Ma et al \citep{Ma_Url11}, whose aim is to detect malicious \ac{URL}s, mainly related with pishing attacks throug e-mailing, but without processing the content or other private data of the user. They extract information from the lexical features (62\% of the total of the gathered features), and also from the host that has the \ac{URL}. It is important to point out that they perform the study over 100 days, and that they work with a quantity of almost 2 million of features. In addition, they implement an online classifier (instead of a batch one), and obtain a 99\% of accuracy. 

% Variable local para emacs, para  que encuentre el fichero maestro de
% compilaci�n y funcionen mejor algunas teclas r�pidas de AucTeX
%%%
%%% Local Variables:
%%% mode: latex
%%% TeX-master: "../Tesis.tex"
%%% End:

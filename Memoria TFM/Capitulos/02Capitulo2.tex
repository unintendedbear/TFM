%---------------------------------------------------------------------
%
%                          Chapter 2
%
%---------------------------------------------------------------------

\chapter{State of the Art}

\begin{FraseCelebre}
\begin{Frase}
...
\end{Frase}
\begin{Fuente}
...
\end{Fuente}
\end{FraseCelebre}

%-------------------------------------------------------------------
\section{State of the Art}
%-------------------------------------------------------------------
\label{cap2:sec:SotA}

The present Master Thesis tries to obtain a URL classification tool for enhancing the security in the client side, as at the end we want to get if a certain URL is secure or not, having as reference a set of rules (derived from a CSP) that allow or deny a set of known \textit{http} requests. For this, Data Mining (DM) and Machine Learning (ML) techniques have been applied. This Chapter gives an overview in a number of solutions given to protect the user, or the company, against unsecure situations.

Due to the nature of the data (URL accesses performed by humans), the used set of data is highly unbalanced \citep{imbalanced_data_05}. In order to deal with this problem there exist several methods in the
literature, but all of them are mainly grouped in three techniques
\citep{imbalance_techniques_02}: 

\begin{itemize}
 \item \textit{Undersampling the over-sized classes}: i.e. reduce the considered number of patterns for the classes with the majority.
 \item \textit{Oversampling the small classes}: i.e. introduce additional (normally synthetic) patterns in the classes with the minority.
 \item \textit{Modifying the cost associated to misclassifying the positive and the negative class} to compensate for the imbalance ratio of the two classes. For example, if the imbalance ratio is 1:10 in favour of the negative class, the penalty of misclassifying a positive example should be 10 times greater.
\end{itemize}

%-------------------------------------------------------------------
\section*{\NotasBibliograficas}
%-------------------------------------------------------------------
\TocNotasBibliograficas

Citamos algo para que aparezca en la bibliograf�a\ldots
\citep{ldesc2e}

\medskip

Y tambi�n ponemos el acr�nimo \ac{CVS} para que no cruja.

Ten en cuenta que si no quieres acr�nimos (o no quieres que te falle la compilaci�n en ``release'' mientras no tengas ninguno) basta con que no definas la constante \verb+\acronimosEnRelease+ (en \texttt{config.tex}).


%-------------------------------------------------------------------
\section*{\ProximoCapitulo}
%-------------------------------------------------------------------
\TocProximoCapitulo

...

% Variable local para emacs, para  que encuentre el fichero maestro de
% compilaci�n y funcionen mejor algunas teclas r�pidas de AucTeX
%%%
%%% Local Variables:
%%% mode: latex
%%% TeX-master: "../Tesis.tex"
%%% End:

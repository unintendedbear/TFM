%---------------------------------------------------------------------
%
%                      resumen.tex
%
%---------------------------------------------------------------------
%
% Contiene el cap�tulo del resumen.
%
% Se crea como un cap�tulo sin numeraci�n.
%
%---------------------------------------------------------------------

\chapter{Resumen}
\cabeceraEspecial{Resumen}

\begin{FraseCelebre}
\begin{Frase}
...
\end{Frase}
\begin{Fuente}
...
\end{Fuente}
\end{FraseCelebre}

Corporate workers increasingly use their own devices for work purposes, in a trend that has come to be called \textit{Bring Your Own Device} (BYOD) philosophy, and companies are starting to include it in their security policies. For this reason, corporate security systems need to be redefined and adapted to these emerging behaviours by the corporate workforce. This work proposes applying soft-computing techniques to improve the security mechanisms inside a company. The actions performed be company workers under a BYOD situation will be treated as events: an action or set of actions yielding to a response. Some of those events might cause a non compliance with some corporate policies, and then it would be necessary to define a set of security rules (action, consequence). Fusthermore, the processing of the extracted knowledge will allow the rules to be adapted.

The goals of this work are the following:

\begin{itemize}
  \item Applying data mining techniques to security-based event data sets in order to detect interesting or suspicious patterns, since they could mean security incidents.
  \item Automattically adapting security rules depending on user behaviour, in order to deal with new security threats. This adaptation will be mainly performed by means of computational intelligence methods.
\end{itemize}

\endinput
% Variable local para emacs, para  que encuentre el fichero maestro de
% compilaci�n y funcionen mejor algunas teclas r�pidas de AucTeX
%%%
%%% Local Variables:
%%% mode: latex
%%% TeX-master: "../Tesis.tex"
%%% End:

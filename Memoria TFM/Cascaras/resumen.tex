%---------------------------------------------------------------------
%
%                      resumen.tex
%
%---------------------------------------------------------------------
%
% Contiene el cap�tulo del resumen.
%
% Se crea como un cap�tulo sin numeraci�n.
%
%---------------------------------------------------------------------

\chapter{Abstract}
\cabeceraEspecial{Abstract}

Corporate workers increasingly use their own devices for work
purposes, in a trend that has come to be called the \textit{Bring Your
  Own Device} (BYOD) philosophy and companies are starting to include
it in their policies. For this reason, corporate security systems need
to be redefined and adapted, by the corporate workforce, to these
emerging behaviours. %adapted _by the workforce_? Why? -JJ
This work proposes applying soft-computing techniques to improve the
security mechanisms % policies m�s que mechanisms, �no? - JJ
inside a company.  The actions performed be company workers under a BYOD situation will be treated as events: an action or set of actions yielding to a response. Some of those events might cause a non compliance with some corporate policies, and then it would be necessary to define a set of security rules (action, consequence). Furthermore, the processing of the extracted knowledge will allow the rules to be adapted.

The goals of this work are the following:

\begin{itemize}
 \item Applying data mining techniques to extract information from
   Company set % company-set security policies or the set of security
               % polices set by the company? - JJ
of Security Policies, and from a set of Internet
   connection patterns of its employees. % No es un objetivo, es un
                                % medio para alcanzar un objetivo - JJ
 \item Combining that information to detect interesting or suspicious
   patterns, since they could lead to security incidents. %interesting
                                %for whom? - JJ
 \item Performing a deep study of the labelled patterns with the Weka
   tool, and find the best classification method and the most
   significant features. % features of what? - JJ
\end{itemize}

All the development of the code, research, and writing of this Thesis has been done on Github \footnote{\url{https://github.com}}. It is open and accesible at: \url{https://github.com/unintendedbear/TFM/}.

\endinput
% Variable local para emacs, para  que encuentre el fichero maestro de
% compilaci�n y funcionen mejor algunas teclas r�pidas de AucTeX
%%%
%%% Local Variables:
%%% mode: latex
%%% TeX-master: "../Tesis.tex"
%%% End:

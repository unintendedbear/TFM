%---------------------------------------------------------------------
%
%                      resumen.tex
%
%---------------------------------------------------------------------
%
% Contiene el cap�tulo del resumen.
%
% Se crea como un cap�tulo sin numeraci�n.
%
%---------------------------------------------------------------------

\chapter{Resumen}
\cabeceraEspecial{Resumen}

Ante la tendencia decreciente de la matriculaci�n de mujeres en las ingenier�as, y a la necesidad no cubierta desde Europa de incorporaci�n de ingenieros e ingenieras al mundo laboral, se plantea un problema a resolver, que es c�mo hacer las ingenier�as m�s atractivas para los y las adolecentes.

Se debe hacer esta distinci�n entre chicos y chicas puesto que los factores por los que finalmente no estudian una ingenier�a son distintos. Esto quiere decir que a ambos sexos les afecta una serie de factores derivados del entorno, de su percepci�n hacia ellos mismos y hacia los diversos aspectos de las ingenier�as, pero que adem�s el hecho de que existan una serie de estereotipos afecta a las chicas por separado.

En este Trabajo Fin de M�ster se presenta un an�lisis de los factores de influencia, y en profundidad de los factores de g�nero, que afectan a la decisi�n de escoger una carrera, en este caso particularizando en las ingenier�as.
Adem�s, se ha observado e identificado la influencia de dichos factores en un contexto de centro secundaria en Granada, en el que se han realizado una serie de encuestas a alumnos de Educaci�n Secundaria Obligatoria, Bahcillerato, y Ciclos Formativos relacionados con la Tecnolog�a y la Inform�tica.

\endinput
% Variable local para emacs, para  que encuentre el fichero maestro de
% compilaci�n y funcionen mejor algunas teclas r�pidas de AucTeX
%%%
%%% Local Variables:
%%% mode: latex
%%% TeX-master: "../Tesis.tex"
%%% End:

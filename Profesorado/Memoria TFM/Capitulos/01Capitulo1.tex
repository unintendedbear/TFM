%---------------------------------------------------------------------
%
%                          Cap�tulo 1
%
%---------------------------------------------------------------------

\chapter{Introducci�n}

\begin{FraseCelebre}
\begin{Frase}
...
\end{Frase}
\begin{Fuente}
...
\end{Fuente}
\end{FraseCelebre}


%-------------------------------------------------------------------
\section{Introducci�n}
%-------------------------------------------------------------------
\label{cap1:sec:introduccion}

Ante la creciente necesidad de ingenieros e ingenieras en Europa \citep{}, los gobiernos europeos se han movilizado para intentar que crezca el inter�s por las ingenier�as en el instituto \citep{Kearney2014}. Lo que es m�s, seg�n la OECD (Organisation for Economic Co-operation and Development) \citep{OECD2006}, la representaci�n femenina en carreras relacionadas con la ciencia y la tecnolog�a permanece debajo del 40\%. En lo referente a Espa�a, seg�n vemos en \ref{ing_total} y en \ref{grado_total}, .

\figuraEx{Bitmap/ing_total.png}{width=0.9\textwidth}{fig:ing_total}%
{Fuente de datos: Ministerio de Educaci�n, Cultura y Deporte. Gobierno de Espa�a.}{Evoluci�n de la}
\figuraEx{Bitmap/grado_total.png}{width=0.9\textwidth}{fig:grado_total}%
{Source: \url{http://www.samsung.com/global/business/mobile/solution/security/samsung-knox}}{Samsung's Knox utility architecture.}

% %-------------------------------------------------------------------
% \section*{\NotasBibliograficas}
% %-------------------------------------------------------------------
% \TocNotasBibliograficas
% 
% Citamos algo para que aparezca en la bibliograf�a\ldots
% \citep{ldesc2e}
% 
% \medskip
% 
% Y tambi�n ponemos el acr�nimo \ac{CVS} para que no cruja.
% 
% Ten en cuenta que si no quieres acr�nimos (o no quieres que te falle la compilaci�n en ``release'' mientras no tengas ninguno) basta con que no definas la constante \verb+\acronimosEnRelease+ (en \texttt{config.tex}).
% 
% 
% %-------------------------------------------------------------------
% \section*{\ProximoCapitulo}
% %-------------------------------------------------------------------
% \TocProximoCapitulo
% 
% ...

% Variable local para emacs, para  que encuentre el fichero maestro de
% compilaci�n y funcionen mejor algunas teclas r�pidas de AucTeX
%%%
%%% Local Variables:
%%% mode: latex
%%% TeX-master: "../Tesis.tex"
%%% End:

%---------------------------------------------------------------------
%
%                          Cap�tulo 2
%
%---------------------------------------------------------------------

\chapter{Estado del Arte}
\label{cap2:EdA}

\begin{FraseCelebre}
\begin{Frase}
...
\end{Frase}
\begin{Fuente}
...
\end{Fuente}
\end{FraseCelebre}


%-------------------------------------------------------------------
\section{Introducci�n}
%-------------------------------------------------------------------
\label{cap1:sec:intro}

Seg�n los resultados resumidos en la secci�n \ref{cap1:sec:factores}, y teniendo en cuenta una de las conclusiones de uno de los estudios analizados \citep{everis2012}, que indica que se podr�a obtener un aumento de la vocaci�n por ense�anzas tecnol�gicas con iniciativas que respondan a las necesidades de los colectivos menos representados en ellas (en este caso, las mujeres), en este cap�tulo se presentan las iniciativas y proyectos encontrados que en la actualidad tratan de impulsar el inter�s de las mujeres por las ingenier�as.


% %-------------------------------------------------------------------
% \section*{\NotasBibliograficas}
% %-------------------------------------------------------------------
% \TocNotasBibliograficas
% 
% Citamos algo para que aparezca en la bibliograf�a\ldots
% \citep{ldesc2e}
% 
% \medskip
% 
% Y tambi�n ponemos el acr�nimo \ac{CVS} para que no cruja.
% 
% Ten en cuenta que si no quieres acr�nimos (o no quieres que te falle la compilaci�n en ``release'' mientras no tengas ninguno) basta con que no definas la constante \verb+\acronimosEnRelease+ (en \texttt{config.tex}).
% 
% 
% %-------------------------------------------------------------------
% \section*{\ProximoCapitulo}
% %-------------------------------------------------------------------
% \TocProximoCapitulo
% 
% ...

% Variable local para emacs, para  que encuentre el fichero maestro de
% compilaci�n y funcionen mejor algunas teclas r�pidas de AucTeX
%%%
%%% Local Variables:
%%% mode: latex
%%% TeX-master: "../Tesis.tex"
%%% End:

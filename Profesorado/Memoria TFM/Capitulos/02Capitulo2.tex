%---------------------------------------------------------------------
%
%                          Cap�tulo 2
%
%---------------------------------------------------------------------

\chapter{Estado del Arte}
\label{cap2:EdA}

\begin{FraseCelebre}
\begin{Frase}
...
\end{Frase}
\begin{Fuente}
...
\end{Fuente}
\end{FraseCelebre}


%-------------------------------------------------------------------
\section{Introducci�n}
%-------------------------------------------------------------------
\label{cap1:sec:intro}

Seg�n los resultados resumidos en la secci�n \ref{cap1:sec:factores}, y teniendo en cuenta una de las conclusiones de uno de los estudios analizados \citep{everis2012}, que indica que se podr�a obtener un aumento de la vocaci�n por ense�anzas tecnol�gicas con iniciativas que respondan a las necesidades de los colectivos menos representados en ellas (en este caso, las mujeres), en este cap�tulo se presentan las iniciativas y proyectos encontrados que en la actualidad tratan de impulsar el inter�s de las mujeres por las ingenier�as.

Las propuestas encontradas son de diversa �ndole y, aunque todas tienen el mismo fin, se pueden clasificar en funci�n de su manera de alcanzarlo.

\begin{itemize}
  \item \textbf{Proyectos de investigaci�n}. Financiados por la Uni�n Europea o por los distintos gobiernos o universidades en distintos pa�ses, con el prop�sito de seguir profundizando en las razones de g�nero por las que la figura de la mujer sigue estando muy poco representada en el mundo de la tecnolog�a y las ingenier�as. Como resultado de esas investigaciones se suelen obtener propuestas que respondan a las necesidades identificadas, pero pocas veces se llegan a implementar.
  \item \textbf{Agrupaciones, comunidades}. Se trata de grupos o ``clubs'' de chicas y mujeres interesadas por la tecnolog�a, que realizan actividades de manera peri�dica y siempre est�n en activo. Pueden estar financiados por empresas, o simplemente de manera altruista.
  \item \textbf{Campamentos, campus y actividades aisladas}. Son actividades parecidas a las que se organizan en los grupos o comunidades, pero de manera puntual. Incluso aunque se realicen cada a�o, se trata de cursos o actividades en un tiempo limitado, tras lo cual no queda una comunidad en activo. Al igual que con las agrupaciones, pueden estar financiadas por empresas o financiarse mediante cuota de inscripci�n.
\end{itemize}

%-------------------------------------------------------------------
\section{Proyectos de investigaci�n}
%-------------------------------------------------------------------
\label{cap2:sec:inv}

El principal fin de los proyectos de investigaci�n o de la investigaci�n en este campo es la identificaci�n de factores de influencia, como los que se han analizado en el Cap�tulo \ref{cap1:intro}, as� como el origen de estos factores, su alcance, y sus consecuencias.

Como se ha mencionado, la investigaci�n suele estar financiada bien por universidades o por las mismas a trav�s de la concesi�n de proyectos. Es el caso del proyecto ``Mind the Gap'' \citep{mtg:site}, financiado en parte por el programa Erasmus+ de la Uni�n Europea. Sus objetivos son mayoritariamente dos:

\begin{itemize}
  \item Ayudar a los profesores a hacer que las chicas se vean atra�das por las STEM.
  \item Inspirar a las chicas para que se vean capaces de estudiar carreras STEM a trav�s de la adquisici�n de algunas capacidades relacionadas con estar carreras y que las anime a sentirse capaces a pesar de enfrentarse a un entorno con mayor�a de presencia masculina.
\end{itemize}

%-------------------------------------------------------------------
\section{Agrupaciones y comunidades}
%-------------------------------------------------------------------
\label{cap2:sec:grupos}

%-------------------------------------------------------------------
\section{Campamentos, campus, actividades aisladas}
%-------------------------------------------------------------------
\label{cap2:sec:campus}



% %-------------------------------------------------------------------
% \section*{\NotasBibliograficas}
% %-------------------------------------------------------------------
% \TocNotasBibliograficas
% 
% Citamos algo para que aparezca en la bibliograf�a\ldots
% \citep{ldesc2e}
% 
% \medskip
% 
% Y tambi�n ponemos el acr�nimo \ac{CVS} para que no cruja.
% 
% Ten en cuenta que si no quieres acr�nimos (o no quieres que te falle la compilaci�n en ``release'' mientras no tengas ninguno) basta con que no definas la constante \verb+\acronimosEnRelease+ (en \texttt{config.tex}).
% 
% 
% %-------------------------------------------------------------------
% \section*{\ProximoCapitulo}
% %-------------------------------------------------------------------
% \TocProximoCapitulo
% 
% ...

% Variable local para emacs, para  que encuentre el fichero maestro de
% compilaci�n y funcionen mejor algunas teclas r�pidas de AucTeX
%%%
%%% Local Variables:
%%% mode: latex
%%% TeX-master: "../Tesis.tex"
%%% End:
